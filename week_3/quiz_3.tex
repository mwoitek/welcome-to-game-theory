% Created 2024-07-13 Sat 12:39
% Intended LaTeX compiler: pdflatex
\documentclass[11pt]{article}
\usepackage[utf8]{inputenc}
\usepackage[T1]{fontenc}
\usepackage{graphicx}
\usepackage{longtable}
\usepackage{wrapfig}
\usepackage{rotating}
\usepackage[normalem]{ulem}
\usepackage{amsmath}
\usepackage{amssymb}
\usepackage{capt-of}
\usepackage{hyperref}
\usepackage[a4paper,left=1cm,right=1cm,top=1cm,bottom=1cm]{geometry}
\usepackage[american, english]{babel}
\usepackage{enumitem}
\usepackage{float}
\usepackage[sc]{mathpazo}
\linespread{1.05}
\renewcommand{\labelitemi}{$\rhd$}
\setlength\parindent{0pt}
\setlist[itemize]{leftmargin=*}
\setlist{nosep}
\date{}
\title{Graded Quiz 3}
\hypersetup{
 pdfauthor={Marcio Woitek},
 pdftitle={Graded Quiz 3},
 pdfkeywords={},
 pdfsubject={},
 pdfcreator={Emacs 29.4 (Org mode 9.8)}, 
 pdflang={English}}
\begin{document}

\thispagestyle{empty}
\pagestyle{empty}
\section*{Problem 1}
\label{sec:org43b4c7f}

\textbf{Answer:} Player 2 does not play R.\\

Player 2 is rational, and Player 1 knows it. So Player 1 knows that, no matter
which strategy he chooses, Player 2 will choose the strategy that maximizes his
payoff. Then consider the best reply by Player 2 for each strategy of Player 1:
\begin{itemize}
\item Player 1 chooses U: Player 2 chooses L;
\item Player 1 chooses B: Player 2 chooses C.
\end{itemize}
Therefore, Player 2 will never respond with strategy R, and Player 1 knows it.
\section*{Problem 2}
\label{sec:org26b23dd}

\textbf{Answer:} Only A.\\

First, consider the possible replies by Player 1 for each strategy of Player 2:
\begin{itemize}
\item Player 2 chooses X:
\begin{itemize}
\item C is the best reply;
\item A is worse;
\item B is the worst reply.
\end{itemize}
\item Player 2 chooses Y:
\begin{itemize}
\item B is the best reply;
\item C is worse;
\item A is the worst reply.
\end{itemize}
\item Player 2 chooses Z:
\begin{itemize}
\item C is the best reply;
\item A is worse;
\item B is the worst reply.
\end{itemize}
\end{itemize}
This makes it clear that C is always a better strategy than A. In other words, A
is dominated by C.\\
Next, consider the possible replies by Player 2 for each strategy of Player 1:
\begin{itemize}
\item Player 1 chooses A:
\begin{itemize}
\item X is the best reply;
\item Y is worse;
\item Z is the worst reply.
\end{itemize}
\item Player 1 chooses B:
\begin{itemize}
\item Z is the best reply;
\item Y is worse;
\item X is the worst reply.
\end{itemize}
\item Player 1 chooses C:
\begin{itemize}
\item Z is the best reply;
\item X is worse;
\item Y is the worst reply.
\end{itemize}
\end{itemize}
In this case, there is no dominated strategy. Therefore, the only dominated
strategy in this game is A.
\section*{Problem 3}
\label{sec:org8d9a2b5}

\textbf{Answer:} Player 2 knows that Player 1 never chooses A. Given that, Z is always
best for Player 2. Therefore, Player 2 chooses Z.\\

Player 1 is rational, and Player 2 knows it. So Player 2 knows that, no matter
which strategy he chooses, Player 1 will choose the strategy that maximizes his
payoff. Then, based on our previous analysis, Player 1 will never select
strategy A. Player 2 is aware of this fact. In other words, Player 2 knows that
his opponent will choose either B or C. In both of these cases, since Player 2
is rational, he will choose the best reply. As we've seen, the best reply to B
is Z. This is also the best strategy against C. Therefore, Player 1 will never
select A, and Player 2 will select Z.
\section*{Problem 4}
\label{sec:org9a38ad3}

\textbf{Answer:} Both players are rational. Both players know that they are both
rational. Both players know that they know that they are both rational.\\

In the previous problem, we've examined the situation in which
\begin{itemize}
\item both players are rational, and
\item Player 2 knows that Player 1 is rational.
\end{itemize}
In this situation (Case 1), Player 1 chooses either B or C, and Player 2 chooses
Z.\\
Next, consider a similar situation (Case 2):
\begin{itemize}
\item both players are rational, and
\item Player 1 knows that Player 2 is rational.
\end{itemize}
Since Player 2 is rational, he'll never play strategy Y. Player 1 is aware of
this fact. This player is rational. So he'll select C, which is the best reply
to both X and Z. Therefore, in this situation, Player 1 chooses C, and Player 2
chooses either X or Z.\\
Let us combine the above conclusions into a single general statement. When both
players are rational and one of them knows about the rationality of his
opponent, the player with more knowledge has only one option, and his opponent
has two options. Then both players knowing that they're both rational isn't
enough for the Nash equilibrium to be played.\\
Finally, consider the following case:
\begin{itemize}
\item both players are rational,
\item both players know that they're both rational, and
\item both players know that they know that they're both rational.
\end{itemize}
First, we start from the perspective of Player 1. This player is rational, and
he knows that Player 2 is also rational. As a result, Player 1 will reach the
same conclusion we did for Case 2. In other words, this player will select
strategy C. But now Player 2 knows this will be his opponent's choice. Since
Player 2 is rational, he'll select his best option, strategy Z. In this case,
the Nash equilibrium is played.\\
To conclude, we now start from Player 2's perspective. This player is rational,
and he knows that his opponent is also rational. As a result, Player 2 will
reach the same conclusion we did for Case 1. In other words, this player will
select strategy Z. But now Player 1 knows this will be his opponent's choice.
Since Player 1 is rational, he'll select his best option, strategy C. Once
again, the Nash equilibrium is played. Notice that, for this to occur, Player 1
has to know that both players know that they're both rational.
\section*{Problem 5}
\label{sec:orgeddb093}

\textbf{Answer:} Strategy B does not provide a strictly higher payoff than Strategy A,
but Strategy A sometimes provides a strictly higher payoff.
\section*{Problem 6}
\label{sec:orge1ff703}

\textbf{Answer:} When there is uncertainty, payoffs represent utility from each
outcome, which describes players' attitude to risks.
\section*{Problem 7}
\label{sec:org0c682e0}

\textbf{Answer:} They do not choose 0 because choosing 1/4 is always better
(irrespective of the opponent's location). Similarly, they do not choose 1
either.
\end{document}
